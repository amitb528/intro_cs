%        File: e.tex
%     Created: Sat Nov 28 04:00 PM 2015 JST
% Last Change: Sat Nov 28 04:00 PM 2015 JST
%
\documentclass[a4paper]{article}
\usepackage[utf8x]{inputenc}
\usepackage[english, hebrew]{babel}
\usepackage{listings}
\usepackage{mathtools}
\lstset{language=Python}
\begin{document}
\section*{שאלה 1}
\subsection*{א}
\subsubsection*{1}
\begin{equation*}
    n(\log_{2}n)^{k}
    = n(\frac {\log_{10}n}{\log_{10}2})^{k} 
    = \frac{1}{(\log_{10}2)^k} n(\log_{10}n)^k 
    = O(n(\log_{10}n)^k)
\end{equation*}
נכון
\subsubsection*{2}
\begin{equation*}
    \begin{split}
        2^{\log_{2}n} = 2^{\frac{\log_{10}n}{\log_{10}2}}
        = (2^{\frac{1}{\log_{10}2}})^{\log_{10}n}
        \geq (2^{3.3})^{\log_{10}n} \\
        = 2^{2.3 \log_{10}n} 2^{\log_{10}n}
        \Rightarrow 2^{\log_2 n} \notin O(2^{\log_{10}n})
    \end{split}
\end{equation*}
לא נכון
\subsubsection*{3}
\begin{eqnarray*}
    f(n) = c\log_2 n \Rightarrow 2^{f(n)} = 2^{c\log_2 n} = n^c
    \Rightarrow 2^{f(n)} \notin O(n)
\end{eqnarray*}
נכון
\subsubsection*{4}
\begin{equation*}
    \sum\limits_{i = 1}^n \sqrt{2^{i}} = \sum\limits_{i = 1}^n (\sqrt{2})^{i}
    = \frac{\sqrt{2} (\sqrt{2}^n - 1)}{\sqrt{2} - 1}
    = \frac{\sqrt{2}}{\sqrt{2} - 1} \sqrt{2}^n - \frac{\sqrt{2}}{\sqrt{2} - 1}
    \in O(\sqrt{2}^n)
\end{equation*}
נכון
\subsection*{ב}
\subsubsection*{1}
הלולאה מבצעת
n
איטרציות. בכל איטרציה מוסיפים לרשימה 
i
איברים. כלומר שהפונקציה מבצעת
$\sum\limits_{i=0}^{n-1} i = \frac{(n-1)n}{2} = \frac{n^2}{2} - \frac{n}{2} = O(n^2)$
\subsubsection*{2}
הלולאה מבצעת
n
איטרציות. בכל איטרציה מוסיפים ללולאה את מספר האיברים שיש ב
tsl
. כלומר שהפונקציה מבצעת
$(n + 2n + 4n + \cdots + 2^{n-1}n) = \sum\limits_{i=0}^{n-1} 2^i n 
= n \sum\limits_{i=0}^{n-1} 2^i
= n (2^n - 1) = O(n 2^n)$
\subsubsection*{3}
הלולאה מבצעת
n
איטרציות. בכל איטרציה יוצרים רשימה חדשה מאיברי הרשימה הנוכחית ועוד רשימת מספרים באורך זהה. כלומר שהפונקציה מבצעת
\begin{equation*}
    \begin{split}
        (2n + 2^2n + 2^3n + \cdots + 2^n n) = \sum\limits_{i=1}^n 2^i n
        = n \sum\limits_{i=1}^n 2^i \\
        = n \sum\limits_{i=0}^n 2^i - 1
        = n (2^{n+1} - 1) = O(n 2^{n+1}) = O(n 2^{n})
    \end{split}
\end{equation*}
\subsubsection*{4}
הלולאה מבצעת
n
איטרציות. בכל איטרציה אנחנו מוסיפים לרשימה רשימה חדשה של איברים בגודל
00001
שלא ממש אכפת לנו מהאיברים שנמצאים בה. כלומר שהפונקציה מבצעת
$ \sum\limits_{i=0}^{n-1} 10000 = 10000n = O(n)$
\subsection*{ג}
בכל איטרציה של הלולאה מתווסף איבר לרשימה
l
. הלולאה ממשיכה כל עוד היא לא הגיעה לאיבר האחרון ברשימה, ולכן היא לעולם לא מסתיימת.
\section*{שאלה 2}
\subsection*{ב}
428.0-
\subsection*{ג}
התוכנית עוצרת כיוון שהיא מבצעת מקסימום מספר סופי של פעולות )עבור הארגומנטים הנתונים(, אך רק לאחר זמן רב )עבור האפסילון הדיפולטי( כיוון שהיא נדרשת לבצע מעל ל
$10^{13}$
פעולות.
fd
\section*{שאלה 3}
\subsection*{א}
בניית הרשימה 
lla
לוקחת
)mn(O
זמן, וזהו גם גודלה.
לולאת ה
elihw
מתבצעת 
)mn(O
פעמים )מוסיפים איבר אחד בכל פעם(. בכל איטרציה  מציאת המינימום היא הפעולה היקרה ביותר וסיבוכיותה
)mn(O
. לכן סה``כ סיבוכיות הפונקציה היא
$O(n^{2}m^{2})$
\subsection*{ד}
הלולאה מבצעת 
m
איטרציות. בכל איטרציה הפונקציה ממזגת את הרשימה הממוזגת שהיא בנתה עד כה עם רשימה נוספת מ
stsl\_fo\_tsl
. כל מיזוג לוקח
)2n + 1n(O
. כלומר שסיבוכיות הפונקציה היא
\begin{equation*}
    \begin{split}
        ( (0+n) + (n+n) + (2n+n) + (3n + n) + \cdots + ( (m-1)n + n)) \\
        = \sum\limits_{i=0}^{m-1} (in+n)
        = n\sum\limits_{i=0}^{m-1} i + \sum\limits_{i=0}^{m-1} n
        = n \frac{(m-1)m}{2} + (m-1)n \\
        = \frac{nm^2}{2} + \frac{nm}{2} - n
        = O(nm^2)
    \end{split}
\end{equation*}
\subsection*{ה}
הסיבוכיות הכללית של הדרך הראשונה היא: \\
$O(nm \log_2 (nm))$ \\
הסיבוכיות הכללית של הדרך השנייה היא: \\
$O(mn \log_2 n + nm^2) = O(mn(\log_2 n + m))$
\subsubsection*{a}
הסיבוכיות של הדרך הראשונה: \\
$O(n \log_2 (cn)) = O(n (\log_2 n + \log_2 c)) = O(n \log_2 n)$ \\
הסיבוכיות של הדרך השנייה: \\
$O(n (\log_2 n + c) = O(n \log_2 n)$ \\
לכן מבחינת סיבוכיות שתי הדרכים עדיפות באותה מידה.
\subsubsection*{b}
הסיבוכיות של הדרך הראשונה: \\
$O(n \log_2 n \log_2(n \log_2 n)) = O(n \log_2 n (\log_2 n + \log_2 \log_2 n)) = O(n \log_2^2 n)$ \\
הסיבוכיות של הדרך השנייה: \\
$O(n \log_2 n (\log_2 n + \log_2 n)) = O(n \log_2 n (2\log_2 n)) = O(n \log_2^2 n)$ \\
לכן מבחינת סיבוכיות שתי הדרכים עדיפות באותה מידה. 
\subsubsection*{c}
הסיבוכיות של הדרך הראשונה: \\
$O(n^2 \log_2 n^2)$ \\
הסיבוכיות של הדרך השנייה: \\
$O(n^2 (\log_2 n + n)) = O(n^3)$ \\
לכן מבחינת סיבוכיות הדרך הראשונה עדיפה.
\section*{שאלה 4}
\subsection*{א}
\begin{equation*}
    2c \log_2 100 = c \log_2 100^2 = c \log_2 10000 \Rightarrow n' = 10000
\end{equation*}
\subsection*{ב}
\begin{equation*}
    2(c \cdot 100) = c \cdot 200 \Rightarrow n' = 200
\end{equation*}
\subsection*{ג}
\begin{equation*}
    2c \cdot 100^2 = c \cdot 20000 = c \cdot \sqrt{2} \cdot 100
    \Rightarrow n' = \lfloor \sqrt{2} \cdot 100 \rfloor = 141 
\end{equation*}
\subsection*{ד}
\begin{equation*}
    2c \cdot 2^{100} = c \cdot 2^{101} \Rightarrow n' = 101
\end{equation*}
\section*{שאלה 6}
\subsection*{א}
ראשית נשים לב כי השורש של 
1 - 000000001**x
ושל
2 - 2**x
הוא 1.
\subsubsection*{1}
לפי חישוב הנגזרת ב0 של
1f
קרובה עד כדי אפסילון ל0 ולכן הגיוני שתודפס ההודעה בשורה
42
. בנוסף השורש של
2f
הוא 1 ולכן הגיוני שקיבלנו
1=toor
. לפיכך האפשרות תיתכן.
\subsubsection*{2}
כיוון שהשורש של הפונקציה
2f
הוא 1 ו
toor
קרוב ל0 האפשרות לא תיתכן.
\subsubsection*{3}
לפי חישוב הנגזרת שמחשב האלגוריתם ב0 של 
1f
היא
100.0
כיוון שאינה קרובה עד כדי אפסילון ל0, לא יתכן כי מודפסת ההודעה בשורה 
42
.
\subsubsection*{4}
כיוון שהשורש של הפונקציה
2f
הוא 1 ו
toor
קרוב ל0 האפשרות לא תיתכן.
\end{document}
