
%%% Local Variables:
%%% mode: latex
%%% TeX-master: t
%%% End:

\documentclass{article}

\usepackage[utf8x]{inputenc}
\usepackage[english, hebrew]{babel}
\usepackage{hebfont}

\begin{document}
\section{שאלה 3}
\subsection{א}

הפונקציה ממירה את הרשימה $words$ לסט, חישוב שלוקח $O(n)$ זמן בהנחה שסט ממומש כמילון. נניח כי מספר המילים השונות הוא $k$ ספירת מספר ההופעות של מילה לוקחת $O(n)$ זמן ומתבצעת $k$ פעמים. במקרה הטוב $k = 1$ ואז סיבוכיות הפונקציה היא $O(n)$ במקרה הרע $k = n$ ואז סיבוכיות הפונקציה היא $O(n^2)$.

במקרה הטוב הפונקציה תבצע כ$5 \cdot 10^4$ פעולות בלבד ולכן תרוץ באלפיות שנייה בודדות.
במקרה הגרוע הפונקציה תבצע כ$2.5 \cdot 10^9$ פעולות ולכן זמן הריצה שלה יהיה בערך $100$ שניות.
)בהנחה שפייתון מבצעת בין $10^7$ ל $10^8$ פעולות בשנייה(

יש בקובץ כ$50000$ מילים. בהנחה שהספר מעניין, נסיק כי המילים שבו מגוונות כמו בספרים אנגלים אחרים ולכן זמן הריצה יהיה קרוב למקרה הגרוע אך נמוך ממנו כיוון שבאנגלית קיימות מילים המופיעות הסתברותית משמעותית יותר ממילים אחרות \\
\L{(the, is, ...)}
 לכן נסיק שהפונקציה תרוץ בערך $10$ שניות.
 
\subsection{ז}

\selectlanguage{english}
\begin{center}
\begin{tabular}{|c|c|}
\hline
the & 2333 \\
\hline
and & 1416 \\
\hline
to & 1200 \\
\hline
he & 1057 \\
\hline
a & 933 \\
\hline
was & 927 \\
\hline
of & 858 \\
\hline
it & 825 \\
\hline
in & 679 \\
\hline
that & 630 \\
\hline
\end{tabular}
\end{center}

\end{document}