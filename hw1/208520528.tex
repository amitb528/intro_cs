\documentclass[12pt]{report}
\usepackage{ucs}
\usepackage[utf8x]{inputenc}
\usepackage[english,hebrew]{babel}
\usepackage{mathtools}
\usepackage{listings}
\usepackage{pgfplots}

\begin{document}
\section*{שאלה 1}
\subsection*{א}
9.

\subsection*{ב}
התכנית מדפיסה את סכום הספרות של המספר

\subsection*{ג}
הפקודה שנכשלה היא ההמרה למספר שלם כיוון שהקלט שהכנסנו הינו ביטוי

\subsection*{ד}
$470$

\subsection*{ה}
נתון ש$num$ הראשון הוא מספר שלם חיובי ומכאן גם אי שלילי. אם כן הערך החדש שמוכנס ל$num$ בכל איטרציה של הלולאה
$\lfloor \frac{num}{10} \rfloor$
גם הוא אי שלילי ולכן הלולאה היא אינסופית.


\section*{שאלה 2}
\subsection*{א}
הפונקציה מקבלת מחרוזת עליה היא עובדת ומחזירה האם המחרוזת היא אלפאנומרית - מכילה רק מספרים ואותיות
\selectlanguage{english}
\lstset{language=Python}

\begin{lstlisting}
str.isalnum("123abc")
-> True

str.isalnum("123-")
-> False
\end{lstlisting}

\selectlanguage{hebrew}
\subsection*{ב}
הפונקציה מקבלת מחרוזת עליה היא עובדת, מחרוזת $sep$ ומספר $maxsplit$ אופציונלי
הפונקציה עוברת על המחרוזת, מחקלת אותה למחרוזות המופרדות ע"י $sep$ ומחזירה רשימה של מחרוזות אלו.
אם הפונקציה מקבלת את $maxsplit$ היא מחקלת את המחרוזת עליה היא עובדת מקסימום $maxsplit$ פעמים.
\selectlanguage{english}
\begin{lstlisting}
str.split("Hello my name is Amit Hi there!", ' ')  
-> ['Hello', 'my', 'name', 'is', 'Amit', 'Hi', 'there!']

str.split("Hello my name is Amit Hi there!", ' ', 2) 
-> ['Hello', 'my', 'name is Amit Hi there!']
\end{lstlisting}

\selectlanguage{hebrew}
\subsection*{ג}
הפונקציה מקבלת מחרוזת עליה היא עובדת ועוד 2 מחרוזות $old$ ו $new$ ומחליפה כל מופע של $old$ במחרוזת עליה היא עובדת ב$new$
\selectlanguage{english}
\begin{lstlisting}
str.replace("Hello Hello Hello".replace, ' ', ',')
-> "Hello,Hello,Hello"

str.replace("Hello! I don't like the word 'Hello'",
'Hello', 'Something')
-> "Something! I don't like the word 'Something'"
\end{lstlisting}

\selectlanguage{hebrew}
\subsection*{ד}
הפונקציה מקבלת מחרוזת עליה היא עובדת ועוד רשימה של מחרוזות
הפונקציה משרשרת את המחרוזות ברשימה יחד עם המחרוזת שהיא עובדת עליה ביניהן ומחזירה את התוצאה.

\selectlanguage{english}
\begin{lstlisting}
str.join(' ', ['hi', 'amit', 'banay'])
-> 'hi amit banay'

str.join("Hello", ["Hi = ", ', Hola = ', ' :)'])
-> 'Hi = Hello, Hola = Hello :)'
\end{lstlisting}

\selectlanguage{hebrew}
\subsection*{ה}
הפונקציה מקבלת מחרוזת עליה היא עובדת ומחרוזת נוספת $s$
הפונקציה מחזירה את מספר ההופעות של $s$ במחרוזת עליה היא עובדת.

\selectlanguage{english}
\begin{lstlisting}
str.count('Hello sdf Hello fds Hello') 
-> 3

str.count('abcda', 'a')
-> 2
\end{lstlisting}

\selectlanguage{hebrew}
\subsection*{ו}
הפונקציה $find$ מקבלת מחרוזת עליה היא עובדת  ומחרוזת נוספת $s$ ומחזירה את האינדקס הנמוך ביותר שבו במחרוזת עליה היא עובדת מופיעה $s$.
אם $s$ לא נמצאת במחרוזת הפונקציה מחזירה $-1$.
הפונקציה $index$ זהה אך במקום להחזיר ערך $-1$ כאשר $s$ לא נמצאת היא זורקת $ValueError$.

\selectlanguage{english}
\begin{lstlisting}
str.find('hellop', 'lo')
-> 3

str.find('hellop', 'a')
-> -1

str.index('hellop', 'lo')
-> 3

str.index('hellop', 'a')
-> ValueError
\end{lstlisting}

\selectlanguage{hebrew}
\subsection*{ז}
הפונקציה $append$ מקבלת רשימה עליה היא עובדת ואובייקט. הפונקציה מוסיפה את האוייקט לסוף הרשימה.
\selectlanguage{english}
\begin{lstlisting}
l = ['a']
list.append(l, 'b')
l == ['a', 'b']
-> True
\end{lstlisting}

\selectlanguage{hebrew}
הפונקציה $extend$ מקבלת רשימה עליה היא עובדת ורשימה נוספת $l$. הפונקציה מוסיפה לרשימה עליה היא עובדת את האיברים ב$l$.
\selectlanguage{english}
\begin{lstlisting}
l1 = ['a']
l2 = ['a', 'b']
l1.extend(l2)
l1 == ['a', 'a', 'b']
-> True
\end{lstlisting}

\selectlanguage{hebrew}
הפונקציה $pop$ מקבלת רשימה עליה היא עובדת ומספר שלם אי שלילי ומחזירה את האובייקט במיקום של האינדקס בתוך הרשימה.
\selectlanguage{english}
\begin{lstlisting}
l = ['a', 'b', 'c']
list.pop(l, 1)
-> 'b'
\end{lstlisting}

\selectlanguage{hebrew}
הפונקציה $remove$ מקבלת רשימה עליה היא עובדת ואובייקט ומוחקת את המופע הראשון שלו ברשימה. אם האובייקט לא נמצא ברשימה הפונקציה זורקת שגיאת $ValueError$
\selectlanguage{english}
\begin{lstlisting}
l = ['a', 'b', 'c']
list.remove(l, 'b')
l == ['a', 'c']
-> True
list.remove(l, 'b')
-> ValueError
\end{lstlisting}

\selectlanguage{hebrew}
הפונקציה $insert$ מקבלת רשימה עליה היא עובדת, אינדקס ואובייקט. הפונקציה מכניסה את האוייקט לתוך הרשימה לפני האיבר שנמצא באינדקס.
\selectlanguage{english}
\begin{lstlisting}
l = ['a', 'b', '1', '2']
list.insert(l, 2, 'c')
l == ['a', 'b', 'c', '1', '2']
True
\end{lstlisting}

\selectlanguage{hebrew}
הפונקציה $sort$ מקבלת רשימה עליה היא עובדת וממיינת אותה.

\selectlanguage{english}
\begin{lstlisting}
l = [1,5,2,4,3]
list.sort(l)
l == [1,2,3,4,5]
-> True
\end{lstlisting}

\selectlanguage{hebrew}
הפונקציה $count$ מקבלת רשימה עליה היא עובדת ואובייקט. הפונקציה מחזירה את כמות ההופעות של האוייקט ברשימה.
\selectlanguage{english}
\begin{lstlisting}
l = [1,2,3,3,4]
list.count(l, 1)
-> 1
list.count(l, 3)
-> 2
\end(lstlisting}

\selectlanguage{english}
\begin{lstlisting}
l = [1,2,3,3,4]
list.index(l, 3)
-> 2

list.index(l, 5)
-> ValueError
\end{lstlisting}

\selectlanguage{hebrew}
הפונקציה $index$ מקבלת רשימה עליה היא עובדת ואובייקט. הפונקציה מחזירה את האינדקס של האובייקט ברשימה עליה היא עובדת. אם האובייקט לא נמצא ברשימה הפונקציה זורקת $ValueError$.
\section*{שאלה ג}

\subsection*{א}
\selectlanguage{english}
f1
\begin{tabular}{| c | c |}
\hline
power & time \\
\hline
200 & 2.2758601289751823e-05 \\
\hline
400 & 5.399589713306341e-05 \\
\hline
800 & 0.00016020270322769647 \\
\hline
1600 & 0.0005551313727210072 \\
\hline
3200 & 0.001958578453240989 \\
\hline
6400 & 0.007663401179115681 \\
\hline
12800 & 0.029707560898259544 \\
\hline
25600 & 0.1169461884028351 \\
\hline
51200 & 0.4659288119876237 \\
\hline
\end{tabular}
f2
\begin{tabular}{| c | c |}
\hline
power & time \\
\hline
200 & 4.0162231016438454e-06 \\
\hline
400 & 7.586199899378698e-06 \\
\hline
800 & 1.204867112392094e-05 \\
\hline
1600 & 2.0527365450107027e-05 \\
\hline
3200 & 3.926974386558868e-05 \\
\hline
6400 & 8.389445156353759e-05 \\
\hline
12800 & 0.0002030424229815253 \\
\hline
25600 & 0.0005631638205159106 \\
\hline
51200 & 0.0018211343513030442 \\
\hline
\end{tabular}
\\

\selectlanguage{hebrew}
שתי הפונקציות גדלות בקצב שגדל ככל שהקלט גדל.
\subsection*{ב}
\selectlanguage{english}
f3
\begin{tabular}{| c | c |}
\hline
power & time \\
\hline
200 & 1.4726153779065498e-05 \\
\hline
400 & 1.7849883363396657e-05 \\
\hline
800 & 2.4097342532058974e-05 \\
\hline
1600 & 3.748475506881732e-05 \\
\hline
3200 & 9.638937021350102e-05 \\
\hline
6400 & 0.00030389426441956857 \\
\hline
12800 & 0.0011446237713244045 \\
\hline
25600 & 0.004342430377647588 \\
\hline
51200 & 0.017179620252676386 \\
\hline
\end{tabular}

\selectlanguage{hebrew}
הפונקציה גדלה בקצב שגדל ככל שהקלט גדל.
היא יעילה בהשוואה לפתרון הראשון אך לא בהשוואה לשני.
\subsection*{ג}
\selectlanguage{english}
f1 with random 1000 digits long numbers
\begin{tabular}{| c | c |}
\hline
number of zeros & time \\
\hline
0 & 0.0021879494552194956 \\
\hline
52 & 0.002345028427953366 \\
\hline
72 & 0.002077726424431603 \\
\hline
81 & 0.0020937913195666624 \\
\hline
90 & 0.0021053937434771797 \\
\hline
100 & 0.002108517473061511 \\
\hline
110 & 0.0021134261914994568 \\
\hline
119 & 0.00218661071266979 \\
\hline
128 & 0.0021839332302988623 \\
\hline
\end{tabular}
\\ 

\selectlanguage{hebrew}
ניתן לראות שמספר האפסים כמעט ואינו משפיע על זמן הריצה.
\selectlanguage{english}
\\
f2 with random 1000 digits long numbers
\begin{tabular}{| c | c |}
\hline
number of zeros & time \\
\hline
0 & 8.166321640601382e-05 \\
\hline
70 & 8.25557108328212e-05 \\
\hline
80 & 0.00010352932349633193 \\
\hline
90 & 8.30019580462249e-05 \\
\hline
100 & 8.30019580462249e-05 \\
\hline
110 & 0.0001129005122493254 \\
\hline
120 & 8.434069877694128e-05 \\
\hline
130 & 8.434069877694128e-05 \\	
\hline
\end{tabular}
\\ 

\selectlanguage{hebrew}
שוב ניתן לראות שמספר האפסים כמעט ואינו משפיע על זמן הריצה.
\selectlanguage{english}
\\
f3 with random 1000 digits long numbers
\begin{tabular}{| c | c |}
\hline
number of zeros & time \\
\hline
74 & 3.6592260585166514e-05 \\
\hline
80 & 2.632857831486035e-05 \\
\hline
90 & 2.722107274166774e-05 \\
\hline
100 & 2.632857831486035e-05 \\
\hline
110 & 2.6774825528264046e-05 \\
\hline
120 & 2.7221071832173038e-05 \\
\hline
131 & 2.7667319045576733e-05 \\
\hline
\end{tabular}
\\

\selectlanguage{hebrew}
שוב ניתן לראות שמספר האפסים כמעט ואינו משפיע על זמן הריצה. במקרה של 
$74$
נראית עליה חריגה בזמן הריצה של הפונקציה.

\subsection*{ד}
הלולאה תיקח המון זמן.
הסיבה לכך היא שהפונקציות הקודמות רצות על כל הספרות של המספר שהן:
$\lceil \log_{10} 2^{200} \rceil$
ואילו הלולאה השנייה רצה 
$2^{100}$
פעמים.

\section*{שאלה 6}
$length = 20, start = 268382, seq = '37579319339117379797'$


\end{document}